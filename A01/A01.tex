\documentclass[11pt]{article}
\usepackage{graphics,graphicx}
\usepackage{amsmath,amssymb}
\usepackage{tabularx}
\usepackage{times}
\usepackage{amssymb}
\usepackage[left=2cm,right=2cm,top=1.cm,bottom=2cm]{geometry}
\setlength{\parskip}{1ex} %--skip lines between paragraphs
\setlength{\parindent}{0pt} %--don't indent paragraphs
\renewcommand{\baselinestretch}{1.2}
\renewcommand{\theequation}{\thesection.\arabic{equation}}

%-- Commands for header
\renewcommand{\title}[1]{\textbf{#1}\\}
\renewcommand{\line}{\begin{tabularx}{\textwidth}{X>{\raggedleft}X}\hline\\\end{tabularx}\\[-0.5cm]}
\newcommand{\leftright}[2]{\begin{tabularx}{\textwidth}{X>{\raggedleft}X}#1%
& #2\\\end{tabularx}\\[-0.5cm]}

% HEADER %
\begin{document}
\title{MATH 2800-01: Mathematics Major Seminar}
\line
\leftright{Assignment \# 01 ~~~~ 14 Sept 2025}{David Haberkorn}
% END HEADER %


\section*{Problem 1}

(a) \space $ |A| $ = 6 \\
(b) \space $ |B| $ = 0 \\
(c) \space $ |C| $ = 3 \\
(d)	\space $ |D| $ = 3 \\
(e) \space $ |E| $ = 10 \\
(f) \space $ |F| $ = $ \infty $ \\


\section*{Problem 2}

(a) \space $ A = \{\cdots, -4, -1, 2, 5, 8, \cdots\} $ = $ \{3n + 2 : n \in \mathbb{Z}\} $ \\
(b) \space $ B = \{\cdots, -5, 0, 5, 10, 15 \cdots\} $ = $ \{5n : n \in \mathbb{Z}\} $ \\
(c) \space $ C = \{ 1, 8, 27, 64, 125 \cdots\} $ = $ \{n^3 : n \in \mathbb{Z}^+\} $ \\


\section*{Problem 3}

Let \\
$ A = \{n \in \mathbb{Z} : 2 \leq |n| < 4\} $ \\
$ B = \{x \in \mathbb{Q} : 2 < x \leq 4\} $ \\
$ C = \{x \in \mathbb{R} : x^2 - (2+\sqrt{2})x + 2\sqrt2 = 0\} $ \\ 
$ D = \{x \in \mathbb{Q} : x^2 - (2+\sqrt{2})x + 2\sqrt2 = 0\} $ \\

(a) \space A = \{-3, -2, 2, 3\} \\
(b) \space Let E be the set $ \{\frac{7}{3}, \frac{8}{3}, \frac{10}{3}\} $. The elements of E are in B, but not in A. \\
(c) \space $ C = \{-2, -\sqrt2\} $ \\
(d) \space $ D = \{-2\} $ \\
(e) \space $ |A| = 4, |C| = 2, |D| = 1 $

\pagebreak

\section*{Problem 4}

For $ A = \{2,3,5,7,8,10,13\} $, let
    $$ B = \{x \in A : x=y+z, where y,z \in A\} $$
    $$ C = \{r \in B : r + s \in B for some s \in B\}. $$
Therefore, 
$B =  \{5, 7, 8, 10, 13\}$ and $C =  \{2, 3, 5, 7, 8, 10\}$ \\


\section*{Problem 5}

(a) $ A = \{1\}, B = \{1,2\}, C = \{1,2,3\} $ \\
(b) $ A = \{1,2,3\}, B = \{A, 7, \pi\}, C = \{\{A, 7, \pi\}, e\} $ \\
(c) $ A = \{\Psi\}, B = \{\kappa, \{\Psi\}, \Omega\}, C = \{17, \Psi\} $


\end{document}
