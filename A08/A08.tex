\documentclass[11pt]{article}
\usepackage{graphics,graphicx}
\usepackage{amsmath,amssymb}
\usepackage{tabularx}
\usepackage{times}
\usepackage{amssymb}
\usepackage[left=2cm,right=2cm,top=1.cm,bottom=2cm]{geometry}
\setlength{\parskip}{1ex} %--skip lines between paragraphs
\setlength{\parindent}{0pt} %--don't indent paragraphs
\renewcommand{\baselinestretch}{1.2}
\renewcommand{\theequation}{\arabic{equation}}


%-- Commands for header
\renewcommand{\title}[1]{\textbf{#1}\\}
\renewcommand{\line}{\begin{tabularx}{\textwidth}{X>{\raggedleft}X}\hline\\\end{tabularx}\\[-0.5cm]}
\newcommand{\leftright}[2]{\begin{tabularx}{\textwidth}{X>{\raggedleft}X}#1%
& #2\\\end{tabularx}\\[-0.5cm]}

% HEADER %
\begin{document}
\title{MATH 2800-01: Mathematics Major Seminar}
\line
\leftright{Assignment \# 08 ~~~~ 10 Dec 2025}{David Haberkorn}
% END HEADER %


\section*{Problem 1}

    \textbf{Result:} Suppose that $ \lim_{x \to a} f(x) = L$, where $L>0$. Then $ \lim_{x \to a} \sqrt{f(x)} = \sqrt{L}$.
    \\
    \textbf{Proof:} Let $\epsilon > 0$, and choose $\delta = \epsilon\sqrt{L}$. Now suppose $0 < |x-a| < \delta$. We have
    \begin{align*}
        |\sqrt{f(x)} - \sqrt{L}| = \frac{|f(x)-L|}{|\sqrt{f(x)} + \sqrt{L}|} \leq \frac{|f(x)-L|}{\sqrt{L}} < \frac{\delta}{\sqrt{L}} = \frac{\epsilon\sqrt{L}}{\sqrt{L}} = \epsilon
    \end{align*}
    Thus, $ \lim_{x \to a} \sqrt{f(x)} = \sqrt{L}$. \hfill $\Box$


\newpage

\section*{Problem 2}

    \textbf{Result:} The function $Q: \mathbb{R} - \{-1\}$ is defined as
    \begin{align*}
        Q(x) = 
        \begin{cases}
        \dfrac{x^2 - 3x + 2}{x^2 - 1}, & x \in \mathbb{R} \setminus \{-1,1\}, \\[10pt]
        -\dfrac{1}{2}, & x = 1.
    \end{cases}
    \end{align*}
    $Q(x)$ is continuous at $x=1$.

    \textbf{Proof:} In order to show continuity at $x=1$, we must show that 
    \begin{enumerate}
        \item $ \lim_{x \to 1} Q(x)$ exists,
        \item $ \lim_{x \to 1} Q(x) = Q(1)$,
        \item $Q(1)$ is defined.
    \end{enumerate}
    From the definition of $Q(x)$ above, we have that $Q(1) = -\frac{1}{2}$, which satisfies the third criteria above. \\
    To show the first and second criteria, we employ an $\epsilon - \delta$ proof. \\
    Let $\epsilon > 0$, and choose $\delta = \min(1, \frac{2}{3}\epsilon)$. Now suppose $ \left| x-1 \right| < 1$.
    \begin{align*}
        \left| x-1 \right| < 1 \\
        -1 < x-1 < 1 \\
        1 < x+1 < 3
    \end{align*}
    This implies that the quantity $\left| x+1 \right| = x+1$. Now suppose $0 < \left|x-1\right| < \delta$. We have
    \begin{align*}
        \left| \frac{x^2-3x+2}{x^2-1} - \frac{-1}{2}\right| = \left| \frac{x-2}{x+1} + \frac{1}{2}\right| = \left| \frac{3(x-1)}{2(x+1)} \right| = \frac{3\left|x-1\right|}{2(x+1)} \leq \frac{3\left|x-1\right|}{2} < \frac{3\delta}{2} \leq \frac{3}{2} \cdot \frac{2}{3}\epsilon = \epsilon
    \end{align*}
    Therefore, the limit exists. \\
    Evaluating $Q(1)$, we have
    \[
    \begin{gathered}
        \frac{x^2-3x+2}{x^2-1} = \frac{x-2}{x+1} \rightarrow \frac{(1)-2}{(1)+1} = -\frac{1}{2}
    \end{gathered}
    \]
    Since the limit both exists and is equal to $Q(1)$, we have satisfied all three necessary criteria. Therefore, the function $Q(x)$ is continuous at $x=1$. \hfill $\Box$


    

\newpage

\section*{Problem 3}

    \textbf{Result:} Suppose two functions, one the bounded function $g:\mathbb{R} \to \mathbb{R}$, meaning there exists a positive real number $B$ such that $\left| g(x) \right| < B$ for each $x \in \mathbb{R}$, and the other $f:\mathbb{R} \to \mathbb{R}$ where $a \in \mathbb{R}$ such that $\lim_{x \to a} f(x) = 0$. Given these two functions, then $\lim_{x \to a} f(x)g(x) = 0$.
    \\
    \textbf{Proof:} Let $\epsilon > 0$, and choose $\delta = \frac{\epsilon}{B}$. Suppose $0 < \left|x-a\right| < \delta$. We have
    \begin{align*}
        \left|f(x)g(x)-0\right| = \left|f(x)g(x)\right| = \left|f(x)\right| \left|g(x)\right| < \left|f(x)\right|B < \delta B = B\cdot \frac{\epsilon}{B} = \epsilon
    \end{align*}
    Therefore, $\lim_{x \to a} f(x)g(x) = 0$.
    \\
    \textbf{Problem:} Use the above result to determine $\lim_{x \to 0} x^2\sin(\frac{1}{x})$.
    \\
    \textbf{Solution:} Let the function $f:\mathbb{R} \to \mathbb{R}$ be defined by $f(x) = x^2$ and let $g:\mathbb{R} \to \mathbb{R}$ be defined by $g(x) = \sin(\frac{1}{x})$. Notice that when evaluating $ \lim_{x \to 0}f(x) = \lim_{x \to 0}x^2 = 0$. Also notice that for all $s \in \mathbb{R}$, $-1 < \sin(s) < 1$ Since $s = \frac{1}{t}$ is a real number for all $ t \in \mathbb{R} - \{0\}$, the function $g(x) = \sin(\frac{1}{x})$ is a bounded function. \\
    From above, when evaluating the limit of a function that evaluates to $0$, and multiplying it by a bounded function, the result is the same limit as the original function. In this example, we have a function (namely $f$) whose limit as $x$ approaches $0$ evaluates to $0$. When multiplying it by the bounded function $g$, we have the same limit as $x$ approaches $0$. Therefore, 
    \begin{align*}
        \lim_{x \to 0}f(x)g(x) = \lim_{x \to 0}x^2\sin(\frac{1}{x}) = 0. 
    \end{align*}
    \hfill $\Box$
    
\newpage


\section*{Problem 4}

    \textbf{Result:} Let $f: [1,\infty) \to [0, \infty)$ be defined by $f(x) = \sqrt{x-1}$. $f$ is both continuous and differentiable at $x=10$.
    \\
    \textbf{Proof:} Let $f: [1,\infty) \to [0, \infty)$ be defined by $f(x) = \sqrt{x-1}$. We will start by showing that $f$ is continuous at $x=10$. \\
    Evaluating $\lim_{x \to 10}\sqrt{x-1}$, we have
    \begin{align*}
        \lim_{x \to 10}\sqrt{x-1} = \sqrt{10-1} = 3
    \end{align*}
    Let $\epsilon > 0$, and choose $\delta = 3\epsilon$. Now suppose $0 < \left|x-10\right| < \delta$. We have
    \begin{align*}
        \left|\sqrt{x-1}-3\right| = \left|\sqrt{x-1}-3\right| \cdot \frac{\left|\sqrt{x-1}+3\right|}{\left|\sqrt{x-1}+3\right|} = \left|\frac{x-10}{\sqrt{x-1}+3}\right|
    \end{align*}
    Since $\sqrt{x-1}$ is always positive (from our domain), we have
    \begin{align*}
        \left|\frac{x-10}{\sqrt{x-1}+3}\right| \leq \left|\frac{x-10}{3}\right| < \frac{\delta}{3} = \frac{3\epsilon}{3} = \epsilon
    \end{align*}
    Because $f$ is defined at $x=10$, and the limit as $x$ approaches $10$ both exists and is equal to $f(10)$, we have that $f$ is continuous. \\
    We now work to show that $f$ is differentiable at $x=10$. Consider the definition of the derivative, where
    \begin{align*}
        f'(a) = \lim_{x \to a} \frac{f(x)-f(a)}{x-a}
    \end{align*}
    exists. From here, we have
    \begin{align*}
        f'(10) = \lim_{x \to 10} \frac{f(x)-f(10)}{x-10} = \lim_{x \to 10} \frac{\sqrt{x-1}-3}{x-10} = \lim_{x \to 10} \frac{\sqrt{x-1}-3}{x-10} \cdot \frac{\sqrt{x-1}+3}{\sqrt{x-1}+3} = \lim_{x \to 10} \frac{(x-10)}{(x-10)(\sqrt{x-1}+ 3)}
    \end{align*}
    Because we are evaluating the limit as $x$ approaches $10$, rather than $x = 10$, we can say that $x \neq 10$ and simplify the $(x-10)$ values from the numerator and denominator.
    \begin{align*}
        \lim_{x \to 10} \frac{(x-10)}{(x-10)(\sqrt{x-1}+ 3)} = \lim_{x \to 10} \frac{1}{\sqrt{x-1}+ 3}
    \end{align*}
    Now, because the function is continuous, we can evaluate the limit as follows.
    \begin{align*}
        \lim_{x \to 10} \frac{1}{\sqrt{x-1}+ 3} = \frac{1}{\sqrt{(10) - 1}+3} = \frac{1}{6}
    \end{align*}
    Since the limit exists, $f$ is differentiable at $x=10$. \hfill $\Box$

\newpage


\end{document}