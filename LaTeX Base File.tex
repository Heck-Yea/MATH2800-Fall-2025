\documentclass[11pt]{article}
\usepackage{graphics,graphicx}
\usepackage{amsmath,amssymb}
\usepackage{tabularx}
\usepackage{times}
\usepackage[left=2cm,right=2cm,top=1.cm,bottom=2cm]{geometry}
\setlength{\parskip}{1ex} %--skip lines between paragraphs
\setlength{\parindent}{0pt} %--don't indent paragraphs
\renewcommand{\baselinestretch}{1.2}
\renewcommand{\theequation}{\thesection.\arabic{equation}}

%-- Commands for header
\renewcommand{\title}[1]{\textbf{#1}\\}
\renewcommand{\line}{\begin{tabularx}{\textwidth}{X>{\raggedleft}X}\hline\\\end{tabularx}\\[-0.5cm]}
\newcommand{\leftright}[2]{\begin{tabularx}{\textwidth}{X>{\raggedleft}X}#1%
& #2\\\end{tabularx}\\[-0.5cm]}
%
% The bulk work of the typesetting begins here %
%
\begin{document}
\title{MATH 2800-01: Mathematics Major Seminar}
\line
\leftright{Assignment \# 01, ~~ 09/27/2030}{John Doe}

\vspace*{-2cm}
\renewcommand{\theequation}{1.\arabic{equation}}
\setcounter{equation}{0}
\section*{}{\bf Problem 1}
An example of PDE is
%
\begin{equation} \label{eq:pde}
	-\Delta u + u_x = 1,
\end{equation}
%
where $-\Delta$ is the usual Laplace operator. In \eqref{eq:pde}, it is usually supplied with a set of boundary conditions.

\newpage
\renewcommand{\theequation}{2.\arabic{equation}}
\section*{} {\bf Problem 2}
Start a problem on a new page. The equation $x^2-2x+1=0$ has only one root which occurs twice.


\end{document}
