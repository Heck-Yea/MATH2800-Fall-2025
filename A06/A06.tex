\documentclass[11pt]{article}
\usepackage{graphics,graphicx}
\usepackage{amsmath,amssymb}
\usepackage{tabularx}
\usepackage{times}
\usepackage{amssymb}
\usepackage[left=2cm,right=2cm,top=1.cm,bottom=2cm]{geometry}
\setlength{\parskip}{1ex} %--skip lines between paragraphs
\setlength{\parindent}{0pt} %--don't indent paragraphs
\renewcommand{\baselinestretch}{1.2}
\renewcommand{\theequation}{\arabic{equation}}


%-- Commands for header
\renewcommand{\title}[1]{\textbf{#1}\\}
\renewcommand{\line}{\begin{tabularx}{\textwidth}{X>{\raggedleft}X}\hline\\\end{tabularx}\\[-0.5cm]}
\newcommand{\leftright}[2]{\begin{tabularx}{\textwidth}{X>{\raggedleft}X}#1%
& #2\\\end{tabularx}\\[-0.5cm]}

% HEADER %
\begin{document}
\title{MATH 2800-01: Mathematics Major Seminar}
\line
\leftright{Assignment \# 06 ~~~~ 13 Nov 2025}{David Haberkorn}
% END HEADER %


\section*{Problem 1}

    \textbf{Result:} The series given by 
    \begin{align}
        1 + 5 + 9 + \cdots + (4n-3) = 2n^2 - n \notag
    \end{align}
    is true for every positive integer n.
    \\
    \textbf{Proof:} We will use induction. Let $n = 1$. We have 
    \begin{align}
        1 =& \ 2(1)^2-1 \notag \\
        1 =& \ 1, \notag
    \end{align}
    so the statement holds for the base case.\\
    Now, suppose that for some integer $k \geq 1$, the formula holds for $n = k$. In other words, the statement
    \begin{align}
        1 + 5 + 9 + \cdots + (4k-3) = 2k^2 - k \notag
    \end{align}
    is true. We now work to show that the following statement is also true.
    \begin{align}
          1 + 5 + 9 + \cdots + (4(k+1)-3) =& \ 2(k+1)^2 - (k+1) \notag \\
          1 + 5 + 9 + \cdots + (4k-3) + (4(k+1)-3) =& \ 2(k+1)^2 - (k+1) \notag \\
          2k^2 - k + (4(k+1)-3) =& \ 2(k+1)^2 - (k+1) \notag \\
          2k^2 + 3k + 1 =& \ 2(k^2+2k+1) - (k+1) \notag \\
          2k^2 + 3k + 1 =& \ 2k^2 + 3k + 1 \notag
    \end{align}
    Thus, by induction, the initial result is true. \hfill $\Box$

\newpage

\section*{Problem 2}

    \textbf{Result:} Let $k$ be a positive integer greater than or equal to 2. Given $k$, then
    \begin{align}
        \frac{k}{k+1} \geq \frac{2}{3}. \notag
    \end{align}
    \textbf{Proof:} Let $k \geq 2$ be an integer. By cross-multiplying the expression above, we have
    \begin{align}
        \frac{k}{k+1} \geq& \ \frac{2}{3} \notag \\
        3k \geq& \ 2(k+1) \notag \\
        3k \geq& \ 2k + 2 \notag \\
        k \geq& \ 2 \notag
    \end{align}
    since $k \geq 2$ from the statement above, the statement has been proven. \hfill $\Box$ \\



    \textbf{Result:} The expression $4^n > n^3$ for every positive integer $n$. 
    \\
    \textbf{Proof:} We will use induction. \\
    Consider the case where $n = 1$. Applying this, we have
    \begin{align}
        4^1 /geq 1^3 \rightarrow 4 /geq 1, \notag
    \end{align}
    which is true. \\
    Now, assume that the statement holds for $n = k$, where $k \geq 2 \in \mathbb{Z}$. Therefore, 
    \begin{align}
        4^k > k^3. \notag
    \end{align}
    We work to show that the statement similarly holds for $n = k+1$. Applying this, we have
    \begin{align}
        4^{(k+1)} =  4 \cdot 4^k >& 4k^3 > (k+1)^3 \notag \\
        4k^3 >& (k+1)^3 \notag \\
        4k^3 >& k^3 + 3k^2 + 3k + 1 \notag \\
        3k^3 >& 3k^2 + 3k + 1 \notag
    \end{align}
    Which is true for all $k \geq 2$. Therefore, by induction, the statement has been proven. \hfill $\Box$

    



\newpage

\section*{Problem 3}

    \textbf{Result:} $7 \mid (3^{4n+1}-5^{2n-1})$ for every positive integer $n$.
    \\
    \textbf{Proof:} We use induction. Consider the case where $n = 1$. We have 
    \begin{align}
        7 \mid& \ (3^{4(1)+1}-5^{2(1)-1}) \notag \\
        7 \mid& \ (3^{5}-5^{1}) \notag \\ 
        7 \mid& \ (243 - 5) \notag \\
        7 \mid& \ 238 \notag
    \end{align}
    which holds because $7 \cdot 34 = 238$. \\
    Now, assume that the hypothesis holds for $n = k$ where $k \geq 2 \in \mathbb{Z}$. From this assumption, and some integer $a$, we have
    \begin{align}
        7 \mid (3^{4k+1} -& \ 5^{2k-1}) \notag \\
        (3^{4k+1} - 5^{2k-1}) &= 7a \notag \\
        3^{4k+1} &= 7a + 5^{2k-1} \notag
    \end{align}
    We now work to show that the same hypothesis holds for $n = k+1$, or in other words,
    \begin{align}
        7 \mid (3^{4(k+1)+1} - 5^{2(k+1)-1}). \notag
    \end{align}
    By definition of division, and some integer $b$, we have
    \begin{align}
        3^{4(k+1)+1} - 5^{2(k+1)-1} &= 3^{4k+5} - 5^{2k+1} \notag \\
        &= 81 \cdot 3^{4k+1} - 25 \cdot 5^{2k-1} \notag \\
        &= 81 \cdot (7a + 5^{2k-1}) - 25 \cdot 5^{2k-1} \notag \\
        &= 81 \cdot 7a + 81 \cdot 5^{2k-1} - 25 \cdot 5^{2k-1} \notag \\
        &= 7(81a) + 56 \cdot 5^{2k-1} \notag \\
        &= 7(81a + 8 \cdot 5^{2k-1}) \notag 
    \end{align}
    Thus, since $81a + 8 \cdot 5^{2k-1}$ is an integer, the desired result has been achieved. \hfill $\Box$

\newpage


\section*{Problem 4}

    \textbf{Result:} $4(k^2+k) < (2k+1)^2$ for every positive integer k.
    \\  
    \textbf{Proof:} Let $k$ be a positive integer. Then, by algebra, we have
    \begin{align}
        4(k^2+k) <& \ (2k+1)^2 \notag \\
        4k^2 + 4k <& \ 4k^2 + 4k + 1 \notag \\
        4k <& \ 4k + 1 \notag \\
        0 <& \ 1 \notag
    \end{align}
    Thus, the statement has been proven. \hfill $\Box$ \\



    \textbf{Result:} $\sum_{m=1}^{n} \frac{1}{\sqrt{m}} \leq 2\sqrt{n} - 1$ for every positive integer $n$.
    \\
    \textbf{Proof:} We use induction. Consider the case where $n = 1$.
    \begin{align*}
        \sum_{m=1}^{1} \frac{1}{\sqrt{m}} &= \frac{1}{\sqrt{1}} = 1 \\
        2\sqrt{n} - 1 &= 2\sqrt{1} - 1 = 2 - 1 = 1
    \end{align*}
    Since $1 \leq 1$, the base case holds. \\
    Now, assume that the hypothesis holds for some positive integer $k \geq 1$. In other words, 
    \begin{align*}
        \sum_{m=1}^{k} \frac{1}{\sqrt{m}} \leq 2\sqrt{k} - 1
    \end{align*} 
    We must show that the hypothesis holds for $n = k+1$. In other words, 
    \begin{align*}
        \sum_{m=1}^{k+1} \frac{1}{\sqrt{m}} \leq 2\sqrt{k+1} - 1
    \end{align*}
    We start with the left-hand side of the equation:
    \begin{align*}
        \sum_{m=1}^{k+1} \frac{1}{\sqrt{m}} &= \left( \sum_{m=1}^{k} \frac{1}{\sqrt{m}} \right) + \frac{1}{\sqrt{k+1}} \leq (2\sqrt{k} - 1) + \frac{1}{\sqrt{k+1}}
    \end{align*}
    To complete the proof, we need to show that $(2\sqrt{k} - 1) + \frac{1}{\sqrt{k+1}} \leq 2\sqrt{k+1} - 1$. By algebra, we have
    \begin{align*}
        (2\sqrt{k} - 1) + \frac{1}{\sqrt{k+1}} &\leq 2\sqrt{k+1} - 1 \\
        2\sqrt{k} + \frac{1}{\sqrt{k+1}} &\leq 2\sqrt{k+1} \\
        \frac{1}{\sqrt{k+1}} &\leq 2\sqrt{k+1} - 2\sqrt{k} \\
        \frac{1}{\sqrt{k+1}} &\leq 2(\sqrt{k+1} - \sqrt{k})
    \end{align*}
    We multiply the right-hand side by the conjugate $\frac{\sqrt{k+1} + \sqrt{k}}{\sqrt{k+1} + \sqrt{k}}$:
    \begin{align*}
        2(\sqrt{k+1} - \sqrt{k}) &= 2 \cdot \frac{(\sqrt{k+1} - \sqrt{k})(\sqrt{k+1} + \sqrt{k})}{\sqrt{k+1} + \sqrt{k}} \\
        &= 2 \cdot \frac{(k+1) - k}{\sqrt{k+1} + \sqrt{k}} \\
        &= \frac{2}{\sqrt{k+1} + \sqrt{k}}
    \end{align*}
    Continuing, we have
    \begin{align*}
        \frac{1}{\sqrt{k+1}} &\leq \frac{2}{\sqrt{k+1} + \sqrt{k}} \\
        \sqrt{k+1} + \sqrt{k} &\leq 2\sqrt{k+1} \\
        \sqrt{k} &\leq \sqrt{k+1}
    \end{align*}
    Since $k$ is a positive integer, $k < k+1$, and thus $\sqrt{k} \leq \sqrt{k+1}$ is true. Therefore, the original inequality holds for $n=k+1$. \\
    By the principle of mathematical induction, the statement $\sum_{m=1}^{n} \frac{1}{\sqrt{m}} \leq 2\sqrt{n} - 1$ holds for every positive integer $n$. \hfill $\Box$



\newpage



\section*{Problem 5}

    \textbf{Result:} A sequence $\{a_n\}$ is defined recursively by $a_1 = 1$, $a_2 = 2$, and $a_n = a_{n-1} + 2a_{n-2}$ for $n \geq 3$. The $n$th term is also defined as $a_n = 2^{n-1}$.
    \\
    \textbf{Proof:} We proceed with strong induction. Consider the cases $n=1$ and $n=2$. For $n=1$, $a_1 = 1$ and $2^{1-1} = 2^0 = 1$. The first base case holds. For $n=2$, $a_2 = 2$ and $2^{2-1} = 2^1 = 2$. The second base case holds. \\
    Now, assume for an arbitrary integer $k \geq 2$ that $a_i = 2^{i-1}$ for all integers $i$ with $1 \leq i \leq k$. \\
    We show that the statement holds for $n=k+1$, i.e., $a_{k+1} = 2^{(k+1)-1} = 2^k$.
    \begin{align*}
    a_{k+1} &= a_k + 2a_{k-1} \\
    &= 2^{k-1} + 2 \cdot 2^{(k-1)-1} \\
    &= 2^{k-1} + 2^{1 + (k-2)} \\
    &= 2^{k-1} + 2^{k-1} \\
    &= 2 \cdot 2^{k-1} \\
    &= 2^{1 + (k-1)}  \\
    &= 2^k
    \end{align*}
    Thus, $a_{k+1} = 2^k$. By the principle of strong mathematical induction, the statement holds for all positive integers $n$. \hfill $\Box$

\newpage


\section*{Problem 6}

    \textbf{Result:} For every integer $n \geq 28$, there exist nonnegative integers $x$ and $y$ such that 
\begin{align*}
    n = 5x + 8y.
\end{align*}

\textbf{Proof:} We use the Strong Principle of Mathematical Induction.

First, we verify the statement for the integers $n = 28, 29, 30, 31,$ and $32$:
\begin{align*}
    28 &= 5(0) + 8(3), \\
    29 &= 5(1) + 8(3), \\
    30 &= 5(6) + 8(0), \\
    31 &= 5(3) + 8(2), \\
    32 &= 5(4) + 8(2).
\end{align*}
Thus, the statement is true for $n = 28, 29, 30, 31,$ and $32$. \\
Now assume that for some integer $k \geq 32$, the statement holds for all integers $m$ satisfying 
\begin{align*}
    28 \leq m \leq k.
\end{align*}
We work to show that the statement holds for $n = k+1$. Since $k \geq 32$, we have
\begin{align*}
    (k+1) - 5 = k - 4 \geq 28.
\end{align*}
By the inductive hypothesis, the integer $k - 4$ can be written as
\begin{align*}
    k - 4 = 5x + 8y
\end{align*}
for some nonnegative integers $x$ and $y$. We add $5$ to both sides:
\begin{align*}
    k+1 &= (k - 4) + 5 \\
        &= 5x + 8y + 5 \\
        &= 5(x+1) + 8y.
\end{align*}
Since $x+1$ and $y$ are nonnegative integers, this expresses $k+1$ in the required form. \\
Therefore, by the Strong Principle of Mathematical Induction, every integer $n \geq 28$ can be written as $5x + 8y$ for some nonnegative integers $x$ and $y$. \hfill $\Box$



\end{document}