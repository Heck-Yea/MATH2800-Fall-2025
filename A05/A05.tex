\documentclass[11pt]{article}
\usepackage{graphics,graphicx}
\usepackage{amsmath,amssymb}
\usepackage{tabularx}
\usepackage{times}
\usepackage{amssymb}
\usepackage[left=2cm,right=2cm,top=1.cm,bottom=2cm]{geometry}
\setlength{\parskip}{1ex} %--skip lines between paragraphs
\setlength{\parindent}{0pt} %--don't indent paragraphs
\renewcommand{\baselinestretch}{1.2}
\renewcommand{\theequation}{\arabic{equation}}


%-- Commands for header
\renewcommand{\title}[1]{\textbf{#1}\\}
\renewcommand{\line}{\begin{tabularx}{\textwidth}{X>{\raggedleft}X}\hline\\\end{tabularx}\\[-0.5cm]}
\newcommand{\leftright}[2]{\begin{tabularx}{\textwidth}{X>{\raggedleft}X}#1%
& #2\\\end{tabularx}\\[-0.5cm]}

% HEADER %
\begin{document}
\title{MATH 2800-01: Mathematics Major Seminar}
\line
\leftright{Assignment \# 04 ~~~~ 15 Oct 2025}{David Haberkorn}
% END HEADER %


\section*{Problem 1}

    \textbf{Result:} The equation $x^5 + 2x - 5 = 0$ has a unique real number solution between $x=1$ and $x=2$.\\
    \\
    \textbf{Proof:} Let $f(x) = x^5 + 2x - 5$. Notice that $f(x)$ is continuous because it is a polynomial, meaning it is continuous in $\mathbb{R}$. \\
    Assume, to the contary, that $f(x)$ has two real number solutions. This implies that there are values $c,d \in (1,2), c \neq d$ where $f(c) = 0$ and $f(d) = 0$. In other words,
    \begin{align}
        f(c) = c^5 + 2c - 5 = 0 \notag \\
        f(d) = d^5 + 2d - 5 = 0 \notag
    \end{align}
    Through algebraic manipulation, we get
    \begin{align}
        c^5 + 2c - 5 =& \ d^5 + 2d - 5 \notag \\
        c^5 + 2c =& \ d^5 + 2d \notag \\
        c^5 - d^5 =& \ 2d - 2c \notag \\
        c^5 - d^5 =& \ 2(d-c) \notag \\
        (c - d)(c^4 + c^3d + c^2d^2 + cd^3 + d^4) =& \ 2(d-c) \notag \\
        (c - d)(c^4 + c^3d + c^2d^2 + cd^3 + d^4) =& \ -2(c-d) \notag \\
        (c - d)(c^4 + c^3d + c^2d^2 + cd^3 + d^4) + 2(c-d)=& \ 0 \notag \\
        (c - d)(c^4 + c^3d + c^2d^2 + cd^3 + d^4 + 2) =& \ 0 \notag
    \end{align}
    In order for the product of the two terms above to equal zero, at least one of $(c-d)$ or $(c^4 + c^3d + c^2d^2 + cd^3 + d^4 + 2)$ must be equal to zero. For the first case, we get
    \begin{align}
        c-d =& 0 \notag \\
        c =& d . \notag
    \end{align}
    For the second case, we get
    \begin{align}
        c^4 + c^3d + c^2d^2 + cd^3 + d^4 + 2 = 0 \notag \\
        c^4 + c^3d + c^2d^2 + cd^3 + d^4 = -2 \notag
    \end{align}
    Notice that since $c,d \in (1,2)$, they are both positive numbers. Also notice that the sum of powers and products of positive numbers must also be a positive number. Since the sum totaling to $-2$ is not possible, the only real values are from before, where we found $c=d$.\\
    However, this is a contradiction since we stated that $c$ and $d$ are unique. Therefore, the equation $x^5 + 2x - 5 = 0$ has only one unique real number solution between $x=1$ and $x=2$. \hfill $\Box$
\newpage

\section*{Problem 2}

    \textbf{Result:} For every positive integer $n \geq 2$, the equation $x^n + (x+1)^n = (x+2)^n$ is false. \\
    \textbf{Disproof (by counterexample):} Let $x = 3$ and $n = 2$. Plugging these values into the anove equation, we arrive at
    \begin{align}
        x^n + (x+1)^n = (x+2)^n \notag\\
        3^2 + (3+1)^2 = (3+2)^2 \notag\\
        3^2 + 4^2 = 5^2 \notag\\
        9 + 16 = 25 \notag\\
        25 = 25 \notag
    \end{align}
    Thus, the statement has been disproven. \hfill $\Box$

\newpage

\section*{Problem 3}

    \textbf{Result:} If $a$ and $b$ are two distinct real numbers, then either $\frac{a+b}{2} > a$ or $\frac{a+b}{2}>b$.\\
    \\
    \textbf{Proof Strategy:} We will use constrapositive. We will work to show that if $\frac{a+b}{2} \leq a$ and $\frac{a+b}{2} \leq b$, then $a$ and $b$ are not distinct, meaning $a=b$. We will work through two cases, then combine the two cases into one result. \hfill $\blacklozenge$\\
    \\
    \textbf{Proof, by contrapositive:} There are two cases we must prove. Let case 1 be $\frac{a+b}{2} \leq a$, and let case 2 be $\frac{a+b}{2} \leq b$, for any two distinct real numbers $a$ and $b$.

    \textit{Case 1}: Let $\frac{a+b}{2} \leq a$, for $a, b \in \mathbb{R}$. Therefore, 
    \begin{align}
        \frac{a+b}{2} & \le a \notag \\
        a+b & \le 2a \notag \\
        b & \le a \label{ineq1}
    \end{align}
    Inequality (1) is our first result.

     \textit{Case 2}: Let $\frac{a+b}{2} \leq b$, for $a, b \in \mathbb{R}$. Therefore, 
    \begin{align}
        \frac{a+b}{2} & \le b \notag \\
        a+b & \le 2b \notag \\
        a & \le b \label{ineq2}
    \end{align}
    Inequality (2) is our second result.
    \textit{Combining cases:} Combining results (1) and (2), we get the dual inequalities 
    $$
        b \leq a \qquad a \leq b
    $$
    For both to be true, then $a$ must equal $b$, which is the result we have been attempting to prove. \hfill $\Box$

\newpage


\section*{Problem 4}

    \textbf{Result:} If $xy$ and $x+y$ are even and $x, y \in \mathbb{Z}$, then both $x$ and $y$ are even.\\
    \\
    \textbf{Proof Strategy:} We will use constrapositive. We will work to show that if $x$ or $y$ is odd, then either $xy$ or $x+y$ is odd. \hfill $\blacklozenge$\\
    \\
    \textbf{Proof, by contrapositive:} Let $x, y \in \mathbb{Z}$, and let $x$ or $y$ be odd. By definition, there exists integers $m$ and $n$ such that $x=2m+1$ and $y=2n+1$. Without loss of generality, we assume that $x$ is odd.\\
    We work to show that $xy$ or $x+y$ is odd when $x$ is odd. $y$ can be even or odd, so we will prove two cases.\\
    \textit{Case 1:} Let $y$ be odd. By definition, $y = 2k+1$ for some $k \in \mathbb{Z}$. Therefore, 
    \begin{align}
        xy = (2m+1)(2k+1) = 4mk+2m+2k+1 = 2(2mk+m+k)+1 \notag
    \end{align}
    By definition, since $2mk+m+k \in \mathbb{Z}$, $xy$ is odd. \\
    \textit{Case 2:} Let $y$ be even. By definition, $y = 2l$ for some $l \in \mathbb{Z}$. Therefore, 
    \begin{align}
        x+y = (2m+1)+(2l) = 2m+2l+1= 2(m+l)+1 \notag
    \end{align}
    By definition, since $m+l \in \mathbb{Z}$, $x+y$ is odd.\\
    Both cases have been proven. \hfill $\Box$

\newpage



\section*{Problem 5}

    \textbf{Result:} For any integer $x$, $3x+1$ is even if and only if $5x-2$ is odd.\\
    \\
    \textbf{Proof Strategy:} We will use two cases, one where $x$ is even, and one where $x$ is odd. We will then prove the biconditional for both cases. \hfill $\blacklozenge$\\
    \\
    \textbf{Proof:} Let $x$ be an integer. We will prove the biconditional with two cases. \\
    \textit{Case 1:} Let $x$ be an even integer. By definition, $x = 2a$ for some integer $a$. Therefore,
    \begin{align}
        3x+1 = 3(2a)+1 = 6a+1 = 2(3a)+1 \notag \\
        5x-2 = 5(2a)-2 = 10a-2 = 2(5a) \notag
    \end{align}
    By definition, since $3a, 5a \in \mathbb{Z}$, then $3x+1$ is odd and $5x-2$ is even.\\
    Since neither $3x+1$ is even, nor $5x-2$ is odd when $x$ is even, we can mark this case irrelevant.\\
    \textit{Case 2:} Let $x$ be an odd integer. By definition, $x = 2b+1$ for some integer $b$. Therefore,
    \begin{align}
        3x+1 = 3(2b+1)+1 = 6b+3+1 = 6b+4 = 2(3b+2) \notag \\
        5x-2 = 5(2b+1)-2 = 10b+5-2 = 10b+2+1 = 2(5b+1)+1 \notag
    \end{align}
    Since $3b+2, 5b+1 \in \mathbb{Z}$, $3x+1$ is even and $5x-2$ is odd.\\
    Since both implications of the biconditional are met when $x$ is odd, and neither implication is met when $x$ is even, the biconditional is satisfied. \hfill $\Box$.

\newpage


\section*{Problem 6}

    \textbf{Result:} For any integer $n$, $5 | n^2$ if and only if $ 5 | n$\\.
    \\
    \textbf{Proof:} We must prove both implications, so we begin by proving that for any integer $n$, $5 | n^2$ if $ 5 | n$. \\
    \textit{Left to Right:} We will prove this implication using contrapositive.\\
    Let $n \in \mathbb{Z}$, and assume that $5 \nmid n $. By definition, there does not exist an integer $a$ such that $n = 5a$, or in other words, $n \neq 5a$.
    \begin{align}
        n \neq& \ 5a \notag \\
        n^2 \neq& \ (5a)^2 \notag \\
        n^2 \neq& \ 25a^2 \notag \\
        n^2 \neq& \ 5(5a^2) \notag
    \end{align}
    Since $n^2$ cannot be written as the product of two integers, $5a^2$ and $5$, $5 \nmid n^2 $. \\
    \textit{Right to left:} We will prove this implication directly.\\
    Let $n \in \mathbb{Z}$, and assume that $5 | n$. By definition, there exists an integer $a$ such that $n = 5a$. Therefore,
    \begin{align}
        n =& 5a \notag\\
        n^2 =& (5a)^2 \notag\\
        n^2 =& 25a^2 \notag\\
        n^2 =& 5(5a^2)
    \end{align}
    Since $n^2$ can be expressed as the product of two integers, $5$ and $5a^2$, $5 | n^2$
    \hfill $\Box$

\newpage



\section*{Problem 7}

    \textbf{Result:} If $a, b \in \mathbb{R}$, then $ab \leq \sqrt{a^2}\sqrt{b^2}$.
    \\
    \textbf{Proof:} Let $a, b \in \mathbb{R}$. Therefore, 
    \begin{align}
        ab \leq& \ \sqrt{a^2}\sqrt{b^2} \notag \\
        ab \leq& \ (a)(b) \notag \\
        ab \leq& \ ab
    \end{align}
    The proof has been satisfied. It should be noted that for the cases where $a, b \leq 0$, we square the values before taking the square root, meaning the expression is still valid for all $a, b, \in \mathbb{R}$.
    \hfill $\Box$

\newpage



\section*{Problem 8}

    \textbf{Result:} Let $a, b \in \mathbb{R}$. If $a>0$ and $b>0$, then $\frac{a}{b}+\frac{b}{a} \geq 2$.
    \\
    \textbf{Proof:} Let $a, b \in \mathbb{R}$. Therefore, 
    \begin{align}
        \frac{a}{b}+\frac{b}{a} \geq& \ 2 \notag \\
        a+\frac{b^2}{a} \geq& \ 2b \notag \\
        a^2+b^2 \geq& \ 2ab \notag \\
        a^2 + b^2 -2ab \geq& \ 0 \notag
    \end{align}
    Now recognize that this inequality is very similar to that outlined in the Law of Cosines, which is pictured in inequality (5).
    \begin{align}
         a^2 + b^2 -2ab\cos(\theta) = c^2 
    \end{align}
    Observe that when the angle, $\theta$, is zero degrees, then the opposite side (length $c$) has a length of zero units. Also observe that $cos(0) = 1$. Therefore,
    \begin{align}
        a^2 + b^2 -2ab\cos(\theta) \geq c^2 \notag \\
        a^2 + b^2 -2ab\cos(0) \geq 0 \notag \\
        a^2 + b^2 -2ab \geq 0 \notag \\
        a^2 + b^2 \geq 2ab \notag
    \end{align}
    The proof has been completed.
    \hfill $\Box$

\newpage
\end{document}