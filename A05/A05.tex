\documentclass[11pt]{article}
\usepackage{graphics,graphicx}
\usepackage{amsmath,amssymb}
\usepackage{tabularx}
\usepackage{times}
\usepackage{amssymb}
\usepackage[left=2cm,right=2cm,top=1.cm,bottom=2cm]{geometry}
\setlength{\parskip}{1ex} %--skip lines between paragraphs
\setlength{\parindent}{0pt} %--don't indent paragraphs
\renewcommand{\baselinestretch}{1.2}
\renewcommand{\theequation}{\arabic{equation}}


%-- Commands for header
\renewcommand{\title}[1]{\textbf{#1}\\}
\renewcommand{\line}{\begin{tabularx}{\textwidth}{X>{\raggedleft}X}\hline\\\end{tabularx}\\[-0.5cm]}
\newcommand{\leftright}[2]{\begin{tabularx}{\textwidth}{X>{\raggedleft}X}#1%
& #2\\\end{tabularx}\\[-0.5cm]}

% HEADER %
\begin{document}
\title{MATH 2800-01: Mathematics Major Seminar}
\line
\leftright{Assignment \# 04 ~~~~ 15 Oct 2025}{David Haberkorn}
% END HEADER %


\section*{Problem 1}

    \textbf{Result:} The equation $x^5 + 2x - 5 = 0$ has a unique real number solution between $x=1$ and $x=2$.\\
    \\
    \textbf{Proof:} Let $f(x) = x^5 + 2x - 5$. Notice that $f(x)$ is continuous because it is a polynomial, meaning it is continuous in $\mathbb{R}$. \\
    Assume, to the contary, that $f(x)$ has two real number solutions. This implies that there are values $c,d \in (1,2), c \neq d$ where $f(c) = 0$ and $f(d) = 0$. In other words,
    \begin{align}
        f(c) = c^5 + 2c - 5 = 0 \notag \\
        f(d) = d^5 + 2d - 5 = 0 \notag
    \end{align}
    Through algebraic manipulation, we get
    \begin{align}
        c^5 + 2c - 5 =& \ d^5 + 2d - 5 \notag \\
        c^5 + 2c =& \ d^5 + 2d \notag \\
        c^5 - d^5 =& \ 2d - 2c \notag \\
        c^5 - d^5 =& \ 2(d-c) \notag \\
        (c - d)(c^4 + c^3d + c^2d^2 + cd^3 + d^4) =& \ 2(d-c) \notag \\
        (c - d)(c^4 + c^3d + c^2d^2 + cd^3 + d^4) =& \ -2(c-d) \notag \\
        (c - d)(c^4 + c^3d + c^2d^2 + cd^3 + d^4) + 2(c-d)=& \ 0 \notag \\
        (c - d)(c^4 + c^3d + c^2d^2 + cd^3 + d^4 + 2) =& \ 0 \notag
    \end{align}
    In order for the product of the two terms above to equal zero, at least one of $(c-d)$ or $(c^4 + c^3d + c^2d^2 + cd^3 + d^4 + 2)$ must be equal to zero. For the first case, we get
    \begin{align}
        c-d =& 0 \notag \\
        c =& d . \notag
    \end{align}
    For the second case, we get
    \begin{align}
        c^4 + c^3d + c^2d^2 + cd^3 + d^4 + 2 = 0 \notag \\
        c^4 + c^3d + c^2d^2 + cd^3 + d^4 = -2 \notag
    \end{align}
    Notice that since $c,d \in (1,2)$, they are both positive numbers. Also notice that the sum of powers and products of positive numbers must also be a positive number. Since the sum totaling to $-2$ is not possible, the only real values are from before, where we found $c=d$.\\
    However, this is a contradiction since we stated that $c$ and $d$ are unique. Therefore, the equation $x^5 + 2x - 5 = 0$ has only one unique real number solution between $x=1$ and $x=2$. \hfill $\Box$

\newpage

\section*{Problem 2}

    \textbf{Result:} For every positive integer $n \geq 2$, the equation $x^n + (x+1)^n = (x+2)^n$ is false. \\
    \textbf{Disproof (by counterexample):} Let $x = 3$ and $n = 2$. Plugging these values into the above equation, we arrive at
    \begin{align}
        x^n + (x+1)^n = (x+2)^n \notag\\
        3^2 + (3+1)^2 = (3+2)^2 \notag\\
        3^2 + 4^2 = 5^2 \notag\\
        9 + 16 = 25 \notag\\
        25 = 25 \notag
    \end{align}
    Thus, the statement has been disproven. \hfill $\Box$

\newpage

\section*{Problem 3}

    \textbf{Result:} If $a \geq 2$ and $b$ are integers, then $a \nmid b$ or $a \nmid b + 1$. \\
    \textbf{Proof, by contradiction:} Assume, to the contrary, that both $a \mid b$ and $a \mid b + 1$, for some $a \geq 2, b, \in \mathbb{Z}$. This implies that there are two integers $s$ and $t$ such that 
    \begin{align}
        b = \ as \notag \\
        b + 1 = \ at \notag
    \end{align}
    Through algebra, we get
    \begin{align}
        b + 1 = \ at \notag \\
        b = \ at + 1 \notag \\
        as = \ at + 1 \notag \\
        1 = a(s-t)
    \end{align}
   Notice that according to equation $(1)$, and since $(s-t) \in \mathbb{Z}$, we get that $a$ divides 1. This implies that $a = \pm 1$, which is a contradiction since we assumed that $a \geq 2$. \\
   Therefore, the result must be true. \hfill $\Box$

\newpage


\section*{Problem 4}

    \textbf{Result:} $\sqrt{3}$ is irrational. \\

    \textbf{Proof, by contradiciton:} Assume, to the contrary, that $\sqrt{3}$ is rational. By definition, $\sqrt{3}$ can be expressed as
    \begin{align}
        \sqrt{3} = \frac{m}{n} \notag
    \end{align}
    for some integers $m, n$ where $n$ is nonzero and the fraction is in its most simplified form. \\
    Through algebra, we get
    \begin{align}
        \sqrt{3} =& \ \frac{m}{n} \notag \\
        3 =& \ \frac{m^2}{n^2} \notag \\
        3n^2 =& \ m^2 
    \end{align}
    The implication of equation $(2)$ is that 3 divides $m^2$, or in other words, $3 \mid m^2$. From the lemma provided, we know that $3 \mid m^2$ if and only if $3 \mid m$. Therefore, we know that $m = 3p$ for some integer $p$. \\
    Again, through algebra, we find
    \begin{align}
        3n^2 =& \ (3p)^2 \notag \\
        3n^2 =& \  9p^2 \notag \\
        n^2 =& \ 3p^2 \notag
    \end{align}
    Without loss of generality, we find that $n = 3q$ for some integer $q$.\\
    However, we have arrived upon a contradiction, since we said that the original fraction $\frac{m}{n} = \frac{3p}{3q}$ is in its most simplified form. Therefore, the original statement must be true. \hfill $\Box$

\newpage



\section*{Problem 5}

    \textbf{Result:} There exist no positive integers $m, n$ such that $m^2 - n^2 = 1$.\\
    
    \textbf{Proof, by contradiction:} Assume, to the contrary, that there do exist two integers $m, n$ such that $m^2 - n^2 = 1$. Through algebra, we get 
    \begin{align}
        3
    \end{align}

\newpage


\section*{Problem 6}

    \textbf{Result:} For any integer $n$, $5 | n^2$ if and only if $ 5 | n$\\.
    \\
    \textbf{Proof:} We must prove both implications, so we begin by proving that for any integer $n$, $5 | n^2$ if $ 5 | n$. \\
    \textit{Left to Right:} We will prove this implication using contrapositive.\\
    Let $n \in \mathbb{Z}$, and assume that $5 \nmid n $. By definition, there does not exist an integer $a$ such that $n = 5a$, or in other words, $n \neq 5a$.
    \begin{align}
        n \neq& \ 5a \notag \\
        n^2 \neq& \ (5a)^2 \notag \\
        n^2 \neq& \ 25a^2 \notag \\
        n^2 \neq& \ 5(5a^2) \notag
    \end{align}
    Since $n^2$ cannot be written as the product of two integers, $5a^2$ and $5$, $5 \nmid n^2 $. \\
    \textit{Right to left:} We will prove this implication directly.\\
    Let $n \in \mathbb{Z}$, and assume that $5 | n$. By definition, there exists an integer $a$ such that $n = 5a$. Therefore,
    \begin{align}
        n =& 5a \notag\\
        n^2 =& (5a)^2 \notag\\
        n^2 =& 25a^2 \notag\\
        n^2 =& 5(5a^2)
    \end{align}
    Since $n^2$ can be expressed as the product of two integers, $5$ and $5a^2$, $5 | n^2$
    \hfill $\Box$

\newpage



\section*{Problem 7}

    \textbf{Result:} If $a, b \in \mathbb{R}$, then $ab \leq \sqrt{a^2}\sqrt{b^2}$.
    \\
    \textbf{Proof:} Let $a, b \in \mathbb{R}$. Therefore, 
    \begin{align}
        ab \leq& \ \sqrt{a^2}\sqrt{b^2} \notag \\
        ab \leq& \ (a)(b) \notag \\
        ab \leq& \ ab
    \end{align}
    The proof has been satisfied. It should be noted that for the cases where $a, b \leq 0$, we square the values before taking the square root, meaning the expression is still valid for all $a, b, \in \mathbb{R}$.
    \hfill $\Box$

\newpage



\section*{Problem 8}

    \textbf{Result:} Let $a, b \in \mathbb{R}$. If $a>0$ and $b>0$, then $\frac{a}{b}+\frac{b}{a} \geq 2$.
    \\
    \textbf{Proof:} Let $a, b \in \mathbb{R}$. Therefore, 
    \begin{align}
        \frac{a}{b}+\frac{b}{a} \geq& \ 2 \notag \\
        a+\frac{b^2}{a} \geq& \ 2b \notag \\
        a^2+b^2 \geq& \ 2ab \notag \\
        a^2 + b^2 -2ab \geq& \ 0 \notag
    \end{align}
    Now recognize that this inequality is very similar to that outlined in the Law of Cosines, which is pictured in inequality (5).
    \begin{align}
         a^2 + b^2 -2ab\cos(\theta) = c^2 
    \end{align}
    Observe that when the angle, $\theta$, is zero degrees, then the opposite side (length $c$) has a length of zero units. Also observe that $cos(0) = 1$. Therefore,
    \begin{align}
        a^2 + b^2 -2ab\cos(\theta) \geq c^2 \notag \\
        a^2 + b^2 -2ab\cos(0) \geq 0 \notag \\
        a^2 + b^2 -2ab \geq 0 \notag \\
        a^2 + b^2 \geq 2ab \notag
    \end{align}
    The proof has been completed.
    \hfill $\Box$

\newpage
\end{document}